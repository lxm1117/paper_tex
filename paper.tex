\documentclass[10pt,letterpaper]{article}
\usepackage[top=0.85in,left=2.75in,footskip=0.75in]{geometry}
%customized
\linespread{1.5}

% amsmath and amssymb packages, useful for mathematical formulas and symbols
\usepackage{amsmath,amssymb}

% Use adjustwidth environment to exceed column width (see example table in text)
\usepackage{changepage}

% Use Unicode characters when possible
\usepackage[utf8x]{inputenc}

% textcomp package and marvosym package for additional characters
\usepackage{textcomp,marvosym}

% cite package, to clean up citations in the main text. Do not remove.
\usepackage{cite}

% Use nameref to cite supporting information files (see Supporting Information section for more info)
\usepackage{nameref,hyperref}

% line numbers
\usepackage[right]{lineno}

% ligatures disabled
\usepackage{microtype}
\DisableLigatures[f]{encoding = *, family = * }

% color can be used to apply background shading to table cells only
\usepackage[table]{xcolor}

% array package and thick rules for tables
\usepackage{array}

% create "+" rule type for thick vertical lines
\newcolumntype{+}{!{\vrule width 2pt}}

% create \thickcline for thick horizontal lines of variable length
\newlength\savedwidth
\newcommand\thickcline[1]{%
  \noalign{\global\savedwidth\arrayrulewidth\global\arrayrulewidth 2pt}%
  \cline{#1}%
  \noalign{\vskip\arrayrulewidth}%
  \noalign{\global\arrayrulewidth\savedwidth}%
}

% \thickhline command for thick horizontal lines that span the table
\newcommand\thickhline{\noalign{\global\savedwidth\arrayrulewidth\global\arrayrulewidth 2pt}%
\hline
\noalign{\global\arrayrulewidth\savedwidth}}

% Remove comment for double spacing
%\usepackage{setspace} 
%\doublespacing

% Text layout
\raggedright
\setlength{\parindent}{0.5cm}
\textwidth 5.25in 
\textheight 8.75in

% Bold the 'Figure #' in the caption and separate it from the title/caption with a period
% Captions will be left justified
\usepackage[aboveskip=1pt,labelfont=bf,labelsep=period,justification=raggedright,singlelinecheck=off]{caption}
\renewcommand{\figurename}{Fig}

% Use the PLoS provided BiBTeX style
\bibliographystyle{plos2015}

% Remove brackets from numbering in List of References
\makeatletter
\renewcommand{\@biblabel}[1]{\quad#1.}
\makeatother

% Leave date blank
\date{}

% Header and Footer with logo
\usepackage{lastpage,fancyhdr,graphicx}
\usepackage{epstopdf}
\pagestyle{myheadings}
\pagestyle{fancy}
\fancyhf{}
\setlength{\headheight}{27.023pt}
\lhead{\includegraphics[width=2.0in]{PLOS-submission.eps}}
\rfoot{\thepage/\pageref{LastPage}}
\renewcommand{\footrule}{\hrule height 2pt \vspace{2mm}}
\fancyheadoffset[L]{2.25in}
\fancyfootoffset[L]{2.25in}
\lfoot{\sf PLOS}

%units
\usepackage{siunitx}
\DeclareSIUnit\Molar{\textsc{m}}
%\sisetup{scientific-notation = true}
%% Include all macros below

\newcommand{\lorem}{{\bf LOREM}}
\newcommand{\ipsum}{{\bf IPSUM}}

%% END MACROS SECTION


\begin{document}
\vspace*{0.2in}

% Title must be 250 characters or less.
\begin{flushleft}
{\Large
\textbf\newline{Simulating detailed Ca\textsuperscript{2+}-calmodulin-CaMKII network in a new way: biophysical attributes that affect CaMKII activation pattern} % Please use "sentence case" for title and headings (capitalize only the first word in a title (or heading), the first word in a subtitle (or subheading), and any proper nouns).
}
\newline
% Insert author names, affiliations and corresponding author email (do not include titles, positions, or degrees).
\\
Ximing Li\textsuperscript{1},
William Holmes\textsuperscript{1}
\\
\bigskip
\textbf{1} Department of Biological Sciences, Neuroscience Program, Ohio University, Athens, Ohio, United States of America
\\
\bigskip

% Insert additional author notes using the symbols described below. Insert symbol callouts after author names as necessary.
% 
% Remove or comment out the author notes below if they aren't used.
%
% Current address notes
%å\textcurrency Current Address: Dept/Program/Center, Institution Name, City, State, Country % change symbol to "\textcurrency a" if more than one current address note
% \textcurrency b Insert second current address 
% \textcurrency c Insert third current address

% Deceased author note
%\dag Deceased

% % Group/Consortium Author Note
% \textpilcrow Membership list can be found in the Acknowledgments section.

% % Use the asterisk to denote corresponding authorship and provide email address in note below.
% * correspondingauthor@institute.edu
\end{flushleft}
% Please keep the abstract below 300 words
\section*{Abstract}
Calcium/calmodulin-dependent protein kinase II (CaMKII) holoenzymes play a critical role in decoding Ca\textsuperscript{2+} signals in neurons. Understanding how this occurs has been the focus of many modeling studies. However, CaMKII is notoriously difficult simulate in detail because of its multi-subunit nature, which causes a combinatorial explosion of the number of species that must be modeled. To study the Ca\textsuperscript{2+}-calmodulin-CaMKII reaction network with detailed kinetics while including the effect of diffusion, we have customized an existing stochastic particle-based simulator Smoldyn to manage the problem of combinatorial explosion. With this new method, spatial and temporal aspects of the signaling network can be studied without compromising biochemical details. We used this new method to examine how calmodulin molecules, both partially loaded and fully loaded with Ca\textsuperscript{2+}, choose pathways to interact with and activate CaMKII under various Ca\textsuperscript{2+} input conditions. We found that the frequency dependence of the CaMKII phosphorylation is intrinsic to the network kinetics and the activation pattern can be modulated by the relative amount of Ca\textsuperscript{2+} to calmodulin and by the rate of Ca\textsuperscript{2+} diffusion. Depending on whether Ca\textsuperscript{2+} influx is saturating or not, calmodulin molecules could choose different routes within the network to activate CaMKII subunits, resulting in different frequency dependence patterns. In addition, the size of the holoenzyme produces a subtle effect on CaMKII activation. The more extended the subunits are organized, the easier for calmodulin molecules to access and activate the subunits. The findings suggest that particular intracellular environmental factors such as crowding and calmodulin availability can play an important role in decoding Ca\textsuperscript{2+} signals and can give rise to distinct CaMKII activation patterns in dendritic spines, Ca\textsuperscript{2+} channel nanodomains and cytoplasm.
% Please keep the Author Summary between 150 and 200 words
% Use first person. PLOS ONE authors please skip this step. 
% Author Summary not valid for PLOS ONE submissions.   
\section*{Author summary}

Ca\textsuperscript{2+} signals are commonly used by cells for various types of activities. In neurons, Ca\textsuperscript{2+} can regulate gene expression and dendritic spine enlargement, strengthen synaptic connectivity, and promote neural growth or even death. These strikingly contrasting cell processes are possible because Ca\textsuperscript{2+} signals of varying strengths and temporal patterns are interpreted differently by intracellular signaling pathways. In our study, we focus on one particular pathway that involves calmodulin and Ca\textsuperscript{2+}/calmodulin-dependent protein kinase II (CaMKII). We have developed a new computational method that avoids the problem of combinatorial explosion inherent with modeling this pathway, and the method allows us to study the pathway with more details than conventional methods. Experiments have shown that CaMKII activity is sensitive to Ca\textsuperscript{2+} signal frequency and our models demonstrate how this frequency dependence relies on Ca\textsuperscript{2+} input amount, calmodulin availability and the Ca\textsuperscript{2+} diffusion rate.

\linenumbers
% Use "Eq" instead of "Equation" for equation citations.
% Camkii binding with F-actin is mostly reported in spines studies, but also seen in cytoplasm as Li Boxing 2016
%Bhalla's paper CaMKII model with diffusion, but simplified CaM states;
% setup with spine geometry; camca4 only; no diffusion effect; stochastic? hybrid? deterministic
\section*{Introduction}
Calcium/calmodulin-dependent protein kinase II (CaMKII) is an important enzyme widely distributed in the central nervous system and other tissues including cardiac muscles~\cite{Wayman:2008gla,Lisman:2002ki,Erickson:2014fs}. CaMKII is involved in intracellular Ca\textsuperscript{2+} signaling as an effector of the Ca\textsuperscript{2+} sensor protein calmodulin (CaM)~\cite{Herring:2016bh,Hell:2014bd,Coultrap:2012ip}. In neurons, CaMKII molecules are essential in long-term potentiation (LTP), which is our best model for the molecular basis of learning and memory~\cite{Bliss:1973jg}. In spines, activated CaMKII molecules interact with postsynaptic density proteins, facilitating actins to reorganize, leading to spine enlargement and upregulation of $alpha$-amino-3-hydroxy-5-methyl-4-isoxazolepropionic acid (AMPA) receptor numbers~\cite{Herring:2016bh}. CaMKII molecules can also phosphorylate AMPA receptors to regulate channel conductance~\cite{Herring:2016bh,Coultrap:2012ip} or form complexes with NMDA receptors~\cite{Hell:2014bd}. In addition to synaptic functions, recent studies~\cite{Ma:2014dr,Li:2016cq} suggest that CaMKII in the soma can act as carriers to shuttle Ca\textsuperscript{2+}-CaM into the nucleus. Ca\textsuperscript{2+}-CaM molecules are then unloaded in the nucleus, participate in the calcium/calmodulin-dependent protein kinase IV (CaMKIV) cascade to activate nuclear transcription factors. Therefore CaMKII molecules play a key role in excitation-transcription coupling. 

The CaMKII molecule is a holoenzyme that consists of 12 subunits grouped in 2 rings~\cite{Lisman:2002ki,Coultrap:2012ip,Stratton:2013el,Hell:2014bd}. Each subunit contains an association domain allowing the formation of multimers, a regulatory domain with CaM binding sites and phosphorylation sites, and a catalytic domain to act as a kinase~\cite{Lisman:2002ki,Coultrap:2012ip,Stratton:2013el,Hell:2014bd}. Activities such as CaM binding and auto-phosphorylation result in various subunit states. For example, in the presence of Ca\textsuperscript{2+}-CaM complexes, each subunit can bind a CaM molecule and expose its catalytic domain to phosphorylate the direct neighbor subunit at its Thr286/Thr287 site. Once phosphorylated, the CaM unbinding rate decreases dramatically, leading to a prolonged activation of the subunit; we say, the CaM molecule is "trapped" by the CaMKII subunit~\cite{Meyer:1992dp}. Even when the CaM molecule unbinds, the CaMKII subunit stays activated, entering an autonomous state. Finally, Thr305/Thr306 can also be phosphorylated. There the threonine site overlap with the CaM binding site and its phosphorylation blocks the binding of CaM molecules. In this case, the subunit becomes "capped". 

Researchers have long been interested in modeling and simulating CaMKII~\cite{Zhabotinsky:2000fp,Holmes:2000uk,Kubota:2001ul,Dupont:2003vq,Bhalla:2004cu,Lucic:2008gt,Zeng:2010bq,Pepke:2010ju,Michalski:2012ds}. However, the structural complexity and multi-state nature of CaMKII present a technical challenge. A major problem is combinatorial explosion~\cite{Stefan:2014gl}. One CaMKII subunit has only a countable number of states to consider, but with 6 or 12 subunits on a holoenzyme, the number of combinations of states for a holoenzyme becomes extremely large. This is a common problem in systems biology since large proteins usually consist of multiple subunits. Software that adopts a rule-based approach such as BioNetGen~\cite{Hlavacek:2006iq} can be used to expand the network based on a set of given reaction rules. But for CaMKII holoenzyme, using BioNetGen to generate the network is still computationally intensive. After expansion with BioNetGen, a 6-subunit holoenzyme with four states for each subunit contains 700 unique species and 12192 unique reactions. The size of the network grows substantially when more detailed kinetics are involved. For example, each CaM has 4 Ca\textsuperscript{2+}-binding sites, giving rise to 9 distinct binding states; each state has its own distinct kinetics to interact with a CaMKII subunit. Then each CaMKII subunit exhibits a distinct phosphorylation rate depending on the state of CaM being bound. For simplicity, consider a 6-subunit CaMKII holoenzyme. Each subunit would potentially have 20 states. Then for the holoenzyme, the total number of unique species reaches over $\frac{6^{20}}{6} \approx 609.36\times10^{12}$. It would be extremely time-consuming and not practical for BioNetGen to generate this network. To overcome this problem, most previous modeling studies of CaMKII have simplified the Ca\textsuperscript{2+}-CaM-CaMKII reaction network by either modeling CaMKII as monomers~\cite{Bhalla:2004cu,Pepke:2010ju} or allowing only CaM fully-loaded with Ca\textsuperscript{2+} to interact with CaMKII subunits~\cite{Zhabotinsky:2000fp,Kubota:2001ul,Dupont:2003vq,Michalski:2012ds}.

Another complication is related to the intracellular environment. Conventionally, biochemical reactions are modeled in a deterministic manner using Ordinary Differential Equations (ODEs). The underlying assumptions are that reactions occur in a spatially homogenous environment, reactants are abundant and not subject to stochasticity, and molecules diffuse sufficiently fast. The majority of previous modeling studies belong to this category except for a few that are stochastic or hybrid models ~\cite{Bhalla:2004cu, Zeng:2010bq, Holmes:2000uk}. However, intracellular space is highly heterogeneous and compartmentalized~\cite{LubyPhelps:2000uj,Dix:2008gy}. Reactions are often restricted to a small space. Structures such as the cytoskeleton, scaffold proteins or endoplasmic reticulum often act as diffusion barriers to slow down molecules. In addition, most previous modeling studies focus on the dendritic spine, where NMDA receptor-channels are the main Ca\textsuperscript{2+} providers. 

In the present work, we focus on the soma where voltage-dependent Ca\textsuperscript{2+} channels (Ca\textsubscript{V}) provide the Ca\textsuperscript{2+} influx, which serves for our future study on modeling Ca\textsuperscript{2+}-CaM signaling through the cytoplasm. Specifically, we examined Ca\textsuperscript{2+}-CaM-CaMKII network activity near L-type Ca\textsuperscript{2+} channels. This is an area less studied in previous models, but nonetheless plays a critical role in excitation-transcription coupling~\cite{Ma:2015bg,Li:2016cq}. To study this Ca\textsuperscript{2+}-CaM-CaMKII signaling network and deal with the problem of combinatorial explosion, we modified the published and freely available simulator Smoldyn~\cite{Andrews:2004fs}. Smoldyn is a particle-based stochastic simulator and has been used for simulating reaction and diffusion processes in cells. It works well for relatively simple reaction networks but not for holoenzymes such as CaMKII. We added new data structures to describe the reaction network in a compact way. CaMKII holoenzymes are modeled as a collection of subunits. Each subunit has a set of binding sites. Subunits react independently and diffuse collectively. Reactions are defined between binding sites. The reaction network is stored in a hash table to allow for lookup during simulations when reactants collide. Therefore expanding and loading a complete network is not required. We tested and verified these modifications by modeling a Ca\textsuperscript{2+}-CaM interaction network and comparing results with COPASI~\cite{Hoops:2006gy}. Moreover, we added detailed components to the network and examined the frequency dependence of CaMKII activation. We found that slow diffusion of Ca\textsuperscript{2+} can dramatically boost CaMKII activation. Also in the presence of a limited number of CaM molecules, CaM must make an intricate choice between binding Ca\textsuperscript{2+} first or a CaMKII subunit first. Interestingly, a change in the decision pattern leads to an altered frequency dependence of the network. 

% % Results and Discussion can be combined.
\section*{Results}
\subsection*{Testing of the simulator modifications}
We first tested the modifications to Smoldyn used in this paper (see Methods) using a Ca\textsuperscript{2+}-CaM network (Fig~\ref{fig1}). The 4 Ca\textsuperscript{2+} binding sites on CaM gives rise to a total of 9 different binding states of a CaM molecule. For the reaction volume we use a $\SI{500}{\nm}\times\SI{500}{\nm}\times\SI{500}{\nm}$ cube with all sides being reflective. The size of this volume mimics that of a Ca\textsuperscript{2+} channel nanodomain or a small dendritic spine head. Initially, the cube contains 3000 Ca\textsuperscript{2+} ions and 700 apoCaM molecules, equivalent to \SI{39.867}{\micro\Molar} and \SI{9.302}{\micro\Molar} respectively. The concentrations are chosen to produce an observable amount of 4-Ca\textsuperscript{2+} bound CaM at the steady state. All molecules are initially uniformly distributed. We tested reactions using two different Ca\textsuperscript{2+} diffusion constants \SI{2.2e-6}{\square\cm\per\s}~\cite{Keller:2008ez} and half of this value \SI{1.1e-6}{\square\cm\per\s}.

We first characterized the reactions in the network to see if results of our stochastic model could reasonably be compared to results with standard ODE methods. A second order chemical reaction in the solvent phase consists of two steps: molecules encountering each other by diffusion followed by the molecules reacting with each other. If the encountering step takes a much longer time to occur than the reacting step, the reaction is diffusion-limited; otherwise, the reaction is activation-limited. Conventionally, an experimentally measured binding kinetic rate, $k_{on}$, can be decomposed into an encounter rate, $k_{enc}$, and an intrinsic activation rate, $k_{a}$, using the following equation~\cite{Rice:198505},
\begin{equation}\frac{1}{k_{on}}=\frac{1}{k_{enc}}+\frac{1}{k_a}\end{equation}
For diffusion-limited reactions, $k_a \gg k_{enc}$ and $k_{on} \approx k_{enc}$; for activation-limited reactions, $k_{on} \approx k_a$ as diffusion is sufficiently fast. Given the diffusion coefficients for Ca\textsuperscript{2+} and CaM and the parameter values for Ca\textsuperscript{2+} and CaM reaction kinetics in Table S1, we calculated the $\frac{k_a}{k_{on}}$ ratios and concluded that the reactions in the network belong to the activation-limited regime. This was also true if diffusion constants were reduced 10 fold. Thus this simple model resembles a well-mixed system and a standard ODE model will provide a good test standard. 
\begin{figure}[!h]
	\caption{{\bf Ca\textsuperscript{2+}-CaM interaction network.}
	(A)Ca\textsuperscript{2+} and CaM interactions can produce 9 different states of CaM. CaM has two lobes, an N lobe and a C lobe, each of which has 2 Ca\textsuperscript{2+} binding sites. Binding sites on the lobes are designated N1, N2, C1 and C2. The N lobe and C lobe bind Ca\textsuperscript{2+} independently of each other, but Ca\textsuperscript{2+} binding within a lobe is cooperative. Thus site N2 cannot bind Ca\textsuperscript{2+} unless N1 has already bound Ca\textsuperscript{2+} and a similar rule applies to sites C2 and C1. In the figure N0C0 represents apoCaM (no Ca\textsuperscript{2+} bound), "N2" represents binding sites N1 and N2 are both bound with Ca\textsuperscript{2+} and "C2" represents similarly that both C1 and C2 have Ca\textsuperscript{2+} bound, "N1" and "C1" represent only the "N1" site and only the "C1" site have Ca\textsuperscript{2+} bound. Arrows represent the direction of binding reactions.
	(B)The same reactions between Ca\textsuperscript{2+} and CaM as shown in A are represented using the syntax of the modified simulator. The binding and unbinding rate constants are labeled at the end of reactions. ‘+’ represents a binding reaction; ‘$\sim$’ represents molecules that are bound. ‘\{\}’ on the left-hand side specifies the binding states of the reactant binding sites or the conditions for a reaction to occur. On the right-hand side, binding sites involved in the reaction are assigned to new values. Notice the "==" sign is used on the left-hand side, but the "=" sign is used on the right-hand side. The "==" represents True or False of an equality, whereas the "=" denotes an assigning operation. 
	}
\label{fig1}
\end{figure}

We coded a deterministic ODE system in COPASI~\cite{Hoops:2006gy} and compared results with our stochastic model using the same kinetics. We found that the stochastic model and the ODE COPASI model exhibited similar time courses for all CaM binding state species. Slowing down Ca\textsuperscript{2+} diffusion did not significantly alter the time courses for the parameter values used. These results suggest that our modifications of Smoldyn are working properly. 
%	 
%\begin{figure}[!h]
%	\caption{{\bf Test of Smoldyn modifications with a simple Ca\textsuperscript{2+}-CaM reaction network.}
%	Plots compare the number of CaM molecules in each of the 9 possible states as computed with the modified stochastic simulator and COPASI. Thick smooth lines are results using COPASI and thin black and red lines are results with the modified simulator for two different values of the calcium diffusion coefficient.
%	}
%\label{fig2}
%\end{figure}

As another verification, we tested the steady state fraction of phosphorylation of CaMKII holoenzymes when they are modeled as multi-subunit complexes. As Michalski and Loew~\cite{Michalski:2012ds} indicated that for a closed system, different subunits number of a holoenzyme results in distinct the steady state fraction of phosphorylated subunits, given a saturating amount of fully loaded CaM molecules and assuming that subunit phosphorylation is allowed only from a neighbor that is CaM bound but not from one that is phosphorylated. In particular, for dimers the steady state fraction of phosphorylated subunits is $\frac{1}{2}$; for trimers, $\frac{2}{3}$; for large subunit numbers, the fraction approaches $1-e^{-1}$. We built a model using a closed system with $\SI{1}{\um}\times\SI{1}{\um}\times\SI{1}{\um}$ geometry. Six thousand CaMKII subunits and 6020 fully loaded CaM were included in the system. Only the following reactions were allowed: bindings and unbindings between CaM and CaMKII, as well as phosphorylations by neighbor subunits that were bound but not phosphorylated. As expected, we obtained steady state phosphorylation levels that match the Michalski and Loew formula.

Finally, using the same closed system, we tested the complete network and compared results with those obtained from a simulation implemented with a spatial Gillespie algorithm~\cite{Zeng:2010bq}. The two methods showed comparable results.

\subsection*{Analysis of the Ca\textsuperscript{2+}-CaM-CaMKII network}

We set up a prototype model to identify the major network branches that contribute to the phosphorylation of CaMKII subunits. The prototype model comprises Ca\textsuperscript{2+}-CaM-CaMKII interactions as shown in Fig~\ref{fig2}A. In this network diagram molecule species are represented as vertices and reactions are represented as edges. CaM molecules and CaMKII holoenzymes with 6-subunit rings were initially uniformly distributed in a $\SI{1}{\um}\times\SI{1}{\um}\times\SI{2}{\um}$ box. As an initial condition, the box contained 6020 apoCaM molecules (\SI{5}{\micro\Molar}) and 12036 CaMKII subunits (\SI{10}{\micro\Molar}), within which 200 CaM molecules are bound to CaMKII. The initial state is at an equilibrium, which we carefully computed by running 5 simulation trials starting with the same numbers of CaM and CaMKII molecules. For simplicity, we did not consider resting intracellular Ca\textsuperscript{2+} are part of the steady state initial condition, as calculations showed its effect to be minor. Ca\textsuperscript{2+} influx was modeled as an entry from a single source at the top center of the box. A previously generated input file having a total of 6 Ca\textsuperscript{2+} bursts delivered at 5 Hz was used to provide Ca\textsuperscript{2+} influx during the simulation (see Methods). We ran the simulation up to \SI{2.5}{\s}, recorded the molecule numbers at every \SI{1}{\ms} and logged all reaction events. Using the events log, we counted the accumulated occurrences of each reaction type at every \SI{10}{\ms}, starting from the initial steady-state until the Ca\textsuperscript{2+} bursts were over. For a particular reaction, the number of occurrences during a time span is considered to be the net number of molecule state changes along the corresponding edge, i.e., the number of unbinding events is subtracted from the number of binding events. A negative number means that unbinding occurs more often than binding within the given time span. 

\begin{figure}[!h]
	\caption{{\bf The complete Ca\textsuperscript{2+}-CaM-CaMKII network.}
	(A)Species are represented by vertices. Edges are labeled to indicate binding reactions. For example, "N1\_rxn" means binding of a Ca\textsuperscript{2+} ion on CaM at the N1 site. "KN1\_rxn" means CaM attached to a CaMKII subunit binds a Ca\textsuperscript{2+} ion on its N1 site. "K-NxCy\_rxn" means a CaMKII subunit binds a CaM that has x Ca\textsuperscript{2+} ions bound on the N-lobe and y Ca\textsuperscript{2+} ions bound on the C-lobe (x and y range from 0 to 2). "Kp-NxCy\_rxn" means a phosphorylated CaMKII subunit binds a CaM with x and y Ca\textsuperscript{2+} ions bound on the N and C lobes respectively. "KpNxCy\_rxn" means phosphorylation of a KNxCy subunit. Red edges mark the predominant pathway. 
	(B)Layers shaded with blue colors. The back layer (Layer1) represents interactions between Ca\textsuperscript{2+} and CaM. The front layer (Layer2) represents interactions between Ca\textsuperscript{2+} and CaM attached to a CaMKII subunit. Layer3 represents interactions between Ca\textsuperscript{2+} and CaM attached to a phosphorylated CaMKII subunit.
	}
\label{fig2}
\end{figure}

We analyzed the edges in the Ca\textsuperscript{2+}-CaM-CaMKII network (Fig~\ref{fig2}A). The network consists of three layers (Fig~\ref{fig2}B). The back layer (Layer1) describes interactions between Ca\textsuperscript{2+} and the 9 states of CaM (NxCy, x,y=0,1,2). The front layer (Layer2) is for reactions of Ca\textsuperscript{2+} with CaM attached to unphosphorylated CaMKII (KNxCy). The middle layer (Layer3) is for reactions of Ca\textsuperscript{2+} with CaM attached to phosphorylated CaMKII (KpNxCy). Starting in Layer1, there are at most 4 possible ways for each CaM species to change state: binding a Ca\textsuperscript{2+} at a C site, binding Ca\textsuperscript{2+} at an N site, binding to an unphosphorylated CaMKII subunit or binding to a phosphorylated CaMKII subunit. However, binding to a phosphorylated CaMKII subunit requires a bound subunit to be phosphorylated first and then lose the bound CaM. This type of reaction rarely occurs during the early stage of the simulation since phosphorylation is slow and CaM unbinding from a phosphorylated subunit is uncommon. Thus we focus on the first 3 types of reactions.

For each CaM state, we counted the accumulated occurrences of the possible reaction types. Results are shown in Fig~\ref{fig3}. The plot reveals a predominant pathway for CaM and CaMKII state transitions as labeled in Fig~\ref{fig2}A. Starting as apoCaM, a CaM molecule tends to bind a Ca\textsuperscript{2+} ion on the C1 site and then on the C2 site, entering the N0C2 state. CaM in the N0C2 state has a strong preference to enter Layer2 by binding to a CaMKII subunit, after which it quickly binds Ca\textsuperscript{2+} ions to the N1 and N2 sites to become KN2C2. Also evident from Fig~\ref{fig3} is that once CaM has two Ca\textsuperscript{2+} ions on either lobe, the preference is to bind to a CaMKII subunit before binding additional Ca\textsuperscript{2+} ions. Once bound to a CaMKII subunit, additional Ca\textsuperscript{2+} ions bind to CaM more quickly. In this scenario, it is rare for CaM to become fully bound with Ca\textsuperscript{2+} ions prior to binding with a CaMKII subunit. Nevertheless, phosphorylation of CaMKII subunits occurs most often when subunits are bound with fully loaded CaM, but still often with CaM having C sites loaded, KN0C2 and KN1C2, as shown in Fig~\ref{fig4}. We note that the way the predominant pathway is chosen at each vertex is a consequence of reaction affinities as given in Table S1 chosen to be consistent with experimental or modeling studies~\cite{Pepke:2010ju,Zeng:2010bq,Lucic:2008gt}. For example, even though N sites bind Ca\textsuperscript{2+} faster, the C sites have higher affinity making Ca\textsuperscript{2+} binding to C sites preferred. 

\begin{figure}[!h]
	\caption{{\bf Accumulated reaction occurrences over time for Layer1 reactions.}
	For a CaM binding state, 3 types of reactions are shown: binding of Ca\textsuperscript{2+} at an N site, binding of Ca\textsuperscript{2+} at a C site or binding of CaM to an unphosphorylated CaMKII subunit. Green arrows indicate reactions involved in the predominant pathway starting from N0C0.
	}
\label{fig3}
\end{figure}

\begin{figure}[!h]
	\caption{{\bf Accumulated reaction occurrences over time for Layer2 reactions.}
	For a CaM binding state, 3 types of reactions are shown: binding of Ca\textsuperscript{2+} at an N site, binding of Ca\textsuperscript{2+} at a C site or phosphorylation of the CaMKII that has the CaM bound.
	}
\label{fig4}
\end{figure}

To confirm the critical role of NxC2 (x=0,1,2) CaM in activation and phosphorylation of CaMKII, we set up two modified reaction schemes (Fig~\ref{fig5}A). In Scheme1, only CaMKII subunits bound with NxC2 are allowed to become phosphorylated. In Scheme2, phosphorylation is allowed only for subunits bound with N2Cx. Note the phosphorylation rates in the two schemes are equivalent, i.e., $k_{on}$ of reaction KpN2C0\_rxn is the same as that of KpN0C2\_rxn, and the $k_{on}$ of KpN2C1\_rxn equals that of KpN1C2\_rxn. Not surprisingly, the two schemes give rise to different phosphorylation levels as shown in Fig~\ref{fig5}B. Scheme 1 performs slightly worse than the whole network, whereas Scheme 2 produces much lower phosphorylation level. Therefore, edges of the network are not equally involved in CaMKII phosphorylation. 

\begin{figure}[!h]
	\caption{{\bf Reaction scheme 1 and 2 adapted from the complete network.}
	(A)In reaction scheme 1, phosphorylation was allowed only from CaMKII bound with NxC2 (blue paths). In reaction scheme 2, phosphorylation was allowed only from CaMKII bound with N2Cx (green paths).
	(B)Comparison of phosphorylation levels with the complete network, scheme 1 and scheme 2. The results of the complete network were summarized from 15 trials of simulations and presented as mean phosphorylation +/- standard deviation.
	} 
\label{fig5}
\end{figure}

\subsection*{CaMKII activation frequency dependence and the detailed Ca\textsuperscript{2+}-CaM-CaMKII network mechanisms}

A classic experiment by DeKonick and Schulman~\cite{DeKoninck:1998wh} showed that \textit{in vitro} CaMKII holoenzyme activation is sensitive to the frequency of Ca\textsuperscript{2+}-CaM pulses. A recent study~\cite{Fujii:2013bg} also showed that \textit{in vivo} glutamate uncaging frequency affects CaMKII activation in dendritic spines. Numerous modeling studies have also reported a dependence of CaMKII activation on the frequency of Ca\textsuperscript{2+} signals~\cite{Pepke:2010ju,Dupont:2003vq,Michalski:2012ds,Kubota:2001ul}. To confirm this CaMKII activation pattern, we tested our 6-subunit CaMKII model using previously generated 10 Hz and 5 Hz Ca\textsuperscript{2+} influx files. The total number of Ca\textsuperscript{2+} ions entering was comparable in the two cases (40433 ions with 5 Hz vs 40435 ions with 10 Hz). As shown in Fig~\ref{fig6}, we observed much higher phosphorylation levels with 10 Hz than with 5 Hz input.

\begin{figure}[!h]
	\caption{{\bf Comparison of 10 Hz and 5 Hz on CaMKII activation.}
	(A)Number of bound CaMKII subunits (including bound and phosphorylated) with 5 Hz (thin line) and 10 Hz (thick line) Ca\textsuperscript{2+} input conditions.
	(B)Number of phosphorylated CaMKII subunits (including bound and phosphorylated) with 5 Hz (thin line) and 10 Hz (thick line) Ca\textsuperscript{2+} input.
	}
\label{fig6}
\end{figure}

We examined the reaction occurrences over time and noticed that compared to 5 Hz input the 10 Hz input results in more bindings between N0C2 CaM and CaMKII (Fig~\ref{fig7}A,B), and consequently more autophosphorylation (Fig~\ref{fig7}C,D). Therefore the frequency effect is inherent to the binding between Ca\textsuperscript{2+}, CaM and CaMKII. However, the frequency effect can be reversed by providing a saturating amount of Ca\textsuperscript{2+}. In Fig~\ref{fig8}, we increased the Ca\textsuperscript{2+} channel number to allow more Ca\textsuperscript{2+} influx per action potential pulse. The total Ca\textsuperscript{2+} influx was again comparable for 5 Hz and 10 Hz. As Ca\textsuperscript{2+} influx was increased, the network produced more phosphorylated CaMKII subunits for both input frequencies, but the phosphorylation level difference between 5 Hz and 10 Hz diminished. Eventually, the 10 Hz input became saturating and the network generated less phosphorylation with 10 Hz than with 5 Hz input (Fig~\ref{fig8}F).
\begin{figure}[!h]
	\caption{{\bf Effects of 10 Hz and 5 Hz on reaction occurrences.} 
	(A)Accumulated reaction occurrences for CaM CaMKII binding reactions in the presence of 5 Hz Ca\textsuperscript{2+} input. 
	(B)The same as in A but for 10 Hz Ca\textsuperscript{2+} input.
	(C)Accumulated reaction occurrences for CaMKII phosphorylation reactions in the presence of 5 Hz Ca\textsuperscript{2+} input. 
	(D)The same as in C but for 10 Hz Ca\textsuperscript{2+} input.
	}
\label{fig7}
\end{figure}

\begin{figure}[!h]
	\caption{{\bf Effects of the various Ca\textsuperscript{2+} influx on frequency dependence.}
	(A-E)Accumulated reaction occurrences for phosphorylation of CaMKII subunits bound with NxC2 in the presence of 5 Hz (thin line) and 10 Hz (thick line) with different levels of Ca\textsuperscript{2+} influx. 
	(F)Summary of phosphorylation levels in the presence of varying amount of Ca\textsuperscript{2+} delivered at 5 Hz and 10 Hz. The advantage of 10 Hz on phosphorylation is reversed as Ca\textsuperscript{2+} influx increases.
	}
\label{fig8}
\end{figure}

Equivalently, a reversal in frequency preference is also seen when available CaM is limited or Ca\textsuperscript{2+} diffusion is slowed (Fig~\ref{fig9}). To demonstrate, we compared simulations using $3.5\times$ Ca\textsuperscript{2+} input with the default CaM (\SI{5}{\micro\Molar}) and limited CaM (\SI{2.5}{\micro\Molar}). Limiting CaM reduced phosphorylation for input at both frequencies. However even though the Ca\textsuperscript{2+} input did not change, limiting CaM resulted in more phosphorylation reactions with 5 Hz than with 10 Hz Ca\textsuperscript{2+} input (Fig~\ref{fig9}A,B). A similar effect can be achieved with slowed Ca\textsuperscript{2+} diffusion. We set Ca\textsuperscript{2+} diffusion as \SI{1.1e-6}{\square\cm\per\s}, which is half of the default speed and does not change the activation-limited regime of the network. Slow diffusion results in a dramatic increase of CaMKII phosphorylation regardless of Ca\textsuperscript{2+} amount and input frequency. However, the frequency preference for 10 Hz is reversed at the $3\times$ Ca\textsuperscript{2+} influx condition (Fig~\ref{fig9}C,D). In other words, with slow diffusion, Ca\textsuperscript{2+} influx increased 3 fold becomes sufficiently saturating to change the frequency preference. It is also worth noting that by combining limited CaM and slow diffusion, reversing the frequency preference is possible with merely twice the amount of Ca\textsuperscript{2+} influx (Fig~\ref{fig9}E,F). 

\begin{figure}[!h]
	\caption{{\bf Effects of CaM availability and Ca\textsuperscript{2+} diffusion on frequency dependence.}
	(A)With $3\times$ Ca\textsuperscript{2+} influx, accumulated reaction occurrences for CaMKII phosphorylation bound with NxC2 in the presence of 5 Hz (thin line) and 10 Hz (thick line) Ca\textsuperscript{2+} input.
	(B)The same as in A except that the total CaM available in the system is decreased to half. 
	(C)The same as in A except that $3.5\times$ Ca\textsuperscript{2+} influx is provided.
	(D)The same as in C except that Ca\textsuperscript{2+} diffusion is slowed to \SI{1.1e-6}{\square\cm\per\s}. 
	(E)The same as in A except that $2\times$ Ca\textsuperscript{2+} influx is provided.
	(F)The same as in E except with both slowed Ca\textsuperscript{2+} diffusion and half the amount of CaM.
	}
\label{fig9}
\end{figure}

To understand why frequency dependence changed, we examined the reaction occurrences over time on Layer1 edges (Fig~\ref{fig10}). We noticed that regardless of the input frequency, the reaction pathway choice deviates from the predominant pathway when Ca\textsuperscript{2+} levels become saturating. In the presence of moderate Ca\textsuperscript{2+} influx (Fig~\ref{fig10}A,B), CaM molecules follow the predominant pathway by choosing reaction paths according to affinities: bindings between N0C2 and CaMKII are more likely to occur than bindings of Ca\textsuperscript{2+} ions on N1 sites of CaM molecules. CaMKII binding with N0C2 dominates all types of CaMKII-CaM binding reactions (Fig~\ref{fig10}D,E,G). However, when Ca\textsuperscript{2+} influx becomes saturating, bindings of Ca\textsuperscript{2+} ions on N1 sites of CaM molecules start to dominate over the bindings between N0C2 and CaMKII (Fig~\ref{fig10}C). CaM molecules tend to stay in Layer1 longer until they get fully loaded with Ca\textsuperscript{2+}. As a result, CaMKII binding with N2C2 increases and becomes dominant over all other CaM CaMKII binding reactions (Fig~\ref{fig10}F,H,I). This decision change gives rise to an altered pathway and eventually a slower rise phosphorylation level. Given the same total amount of Ca\textsuperscript{2+}, 10 Hz influx saturates CaM in a shorter time than 5 Hz. Correspondingly, the switch of pathway choice is more dramatic for 10 Hz at $4.5\times$ Ca\textsuperscript{2+}, resulting in the reversed frequency preference as observed previously. 

\begin{figure}[!h]
	\caption{{\bf Change of pathway choice as Ca\textsuperscript{2+} influx increases}
	(A)With $2\times$ Ca\textsuperscript{2+} input, reaction occurrences of CaM N0C2 binding Ca\textsuperscript{2+} at the N1 site (blue) and directly binding with CaMKII (black). Thick lines are for 10 Hz input, and thin lines are for 5 Hz input. 
	(B)The same as in A except that $3\times$ Ca\textsuperscript{2+} influx is given.
	(C)The same as in A except that $4.5\times$ Ca\textsuperscript{2+} influx is given.
	(D-F)Accumulated reaction occurrences for CaM CaMKII binding reactions with varying amount of total Ca\textsuperscript{2+} influx from $2\times$, $3\times$ to $4.5\times$. With Ca\textsuperscript{2+} input delivered at 5 Hz. 
	(G-I)The same as in D-F except that Ca\textsuperscript{2+} input is delivered at 10 Hz.
	}
\label{fig10}
\end{figure}

To validate our conclusion about the pathway decision change, we used a reduced reaction network to capture the observed frequency preference reversal (Fig~\ref{fig11}). The reduced network is derived from the initial steps in the whole Ca\textsuperscript{2+}-CaM-CaMKII network. We used the original Smoldyn to simulate the simple network, in order to demonstrate that the reversal of frequency preference is not an artifact of our modified simulator but is inherent to the network itself. We used 5 pulses of instantaneous Ca\textsuperscript{2+} release as the input and varied the amount of Ca\textsuperscript{2+} influx per pulse. The CaMKII molecules are modeled as monomers with an arbitrary phosphorylation rate of \SI{1}{\per\s}. As expected, the reversal of frequency preference can be qualitatively captured by the simple reaction scheme (Fig~\ref{fig11}A). The 10 Hz stimulus becomes saturating and fails to generate more phosphorylation than 5 Hz when Ca\textsuperscript{2+} influx reaches 40000 ions per pulse (Fig~\ref{fig11}B).

\begin{figure}[!h]
	\caption{{\bf Change of pathway choice demonstrated using a reduced network}
	(A)The reduced reaction network derived from the complete network. CaMKII is modeled as a monomer and undergoes phosphorylation at an arbitrary rate \SI{1}{\per\s} as long as bound with a CaM N1C2. 
	(B)Summary of phosphorylation levels in the presence of varying Ca\textsuperscript{2+} ions per pulse from 10000 ions/pulse to 50000 ions/pulse delivered at 5 Hz and 10 Hz. Data from 5 trials are summarized.
	}
\label{fig11}
\end{figure}


\subsection*{The structural organization of CaMKII subunits}

Recent work on CaMKII holoenzyme structure suggests that how subunits are organized can affect the activation of the holoenzyme. In particular, whether subunits are arranged in a compact or an extended way can change the accessibility of CaM to CaMKII. It is known that the structural arrangement is related to a linker region between a subunit's kinase domain and the holoenzyme central hub. Bayer et al.~\cite{Bayer:2002er} examined splice variants of $\beta$-CaMKII, which have identical kinase domains yet different linker lengths. These variants responded to Ca\textsuperscript{2+} oscillations differently, even though they showed no response difference to prolonged Ca\textsuperscript{2+}-CaM input. In particular, for Ca\textsuperscript{2+} pulse input, the variants with longer linker length exhibited a higher autophosphorylation rate. Another study by Chao et al.~\cite{Chao:2011iw} also indicated that the linker length affects the capability of a subunit's kinase domain to undock from the central hub. Undocking helps the subunit to release from an autoinhibited state. Thus a longer linker is expected to keep the kinase domain further away from the central hub, allowing a better chance for Ca\textsuperscript{2+}-CaM to access the binding site.

In our prototype model, holoenzyme subunits have distinct physical locations and are separated by \SI{8}{\nm} from its neighbors. To examine the effect of a longer linker length on CaMKII activation, we varied the radius of a one-ring 6-subunit holoenzyme from \SI{5}{\nm} to \SI{15}{\nm}. Neighbor subunits are equally spaced at the same distance as the holoenzyme radius. Studies suggest that a typical one-ring holoenzyme radius is within \SI{5}{\nm} to \SI{8}{\nm}~\cite{Chao:2011iw,Gaertner:2004jk}. For each radius condition, we ran 15 simulation trials using the default Ca\textsuperscript{2+} influx delivered at 5 Hz. As shown in Fig~\ref{fig12}, there is a slight but distinguishable increase in the phosphorylation level as the radius of the holoenzyme increases.

\begin{figure}[!h]
	\caption{{\bf Effects of holoenzyme size on CaMKII phosphorylation.}
	Phosphorylation levels as a function of the holoenzyme radius presented as mean + standard deviation. Each radius condition is summarized from 15 trials.
	}
\label{fig12}
\end{figure}


\section*{Discussion}
In the study, we present an efficient approach to simulate multi-subunit molecules with detailed kinetics in reaction-diffusion networks. Our approach is an adaptation of Smoldyn~\cite{Andrews:2004fs}, a particle-based stochastic simulator that allows reaction and diffusion to be simulated in spatially heterogenous environments. Our adaptation adds new data structures to Smoldyn to describe reactions at the binding site level, leading to an efficient solution of the problem of combinatorial explosion inherent in models with multi-subunit molecules~\cite{Stefan:2014gl}. We also introduced an intuitive approach to analyzing the pathway choices of a network based on the reaction history of a simulation, allowing us to grasp insights quickly about how reactions procced in a large network. In traditional analyses, the numbers of each molecular species are examined to resolve the dynamics of the network. However this procedure is indirect at best when it comes to drawing insights about network behavior. Furthermore, being able to track reaction history of a system is not possible in deterministic simulations. Here we were able to count the number of times a reaction occurs easily and we demonstrated that this reaction history can be highly informative and can allow us to see clearly how the favored network pathway may change with different input conditions.

We used the modified Smoldyn simulator and reaction history information to obtain several insights into the Ca\textsuperscript{2+}-CaM-CaMKII reaction network. First, under physiological conditions when Ca\textsuperscript{2+} influx is low to moderate, CaM molecules partially loaded with Ca\textsuperscript{2+} are important for CaMKII activation. In particular, reaction history shows that CaM molecules that have 2 Ca\textsuperscript{2+} ions attached on the C lobe (in particular species N0C2, but also N1C2 and N2C2) preferentially bind to CaMKII subunits before adding additional Ca\textsuperscript{2+} ions to the N lobe. This is consistent with the predominant pathway hypothesis suggested by Pepke et al.~\cite{Pepke:2010ju} as well as with experimental work by Shifman et al.~\cite{Shifman:2006hw}. Nevertheless, phosphorylation was found to occur primarily from CaMKII bound with CaM fully loaded with Ca\textsuperscript{2+}(KN2C2) and higher frequencies of Ca\textsuperscript{2+} input resulted in more CaMKII phosphorylation.

Second, while CaMKII activation is known to be sensitive to the frequency of Ca\textsuperscript{2+} signals~\cite{DeKoninck:1998wh}, we found that the frequency dependence is reversed with stronger Ca\textsuperscript{2+} signals with more CaMKII subunit phosphorylation seen at 5 Hz input than at 10 Hz. Reaction history shows that this occurs because of a change in the predominant pathway for CaMKII activation in which CaM with 2 Ca\textsuperscript{2+} ions on the C lobe now becomes more likely to bind Ca\textsuperscript{2+} on the N lobe than to bind with CaMKII. This is significant because many experiments that study frequency dependence are done in conditions where CaM is saturated with Ca\textsuperscript{2+} and this is rarely the situation encountered by the cell.

Third, we found that factors such as CaM availability and Ca\textsuperscript{2+} diffusion can also affect the frequency dependence of CaMKII activation by Ca\textsuperscript{2+} signals, also by changing the predominant pathway for CaMKII activation. A limited amount of CaM makes the given Ca\textsuperscript{2+} input more likely to saturate available CaM on both lobes before binding to CaMKII. Similarly, slow Ca\textsuperscript{2+} diffusion allows more extensive interactions between Ca\textsuperscript{2+} and CaM thus making it more likely for a given amount of Ca\textsuperscript{2+} input to become saturating. Experimental studies suggest that the number of freely diffusible CaM molecules is highly limited \textit{in vivo}~\cite{2008BpJ....95.6002S,Persechini:2002tb,LubyPhelps:1995kl} and limited CaM further implies a regulatory role for the many endogenous CaM binding proteins~\cite{Skene:1990kf}. A limited amount of CaM or a slowed Ca\textsuperscript{2+} diffusion may permit enough Ca\textsuperscript{2+} to bind to available CaM and allow CaMKII activation to occur at lower frequency Ca\textsuperscript{2+} signals.

Fourth, it is known that intracellular crowding and spatial homogeneity can slow down molecule diffusion. For example, a recent biophysical study~\cite{2013PNAS..11015794T} suggests that the diffusion of Ca\textsuperscript{2+} ions can be reduced by ten times in a nanodomain around the Ca\textsuperscript{2+} channel mouth. We believe that such a restriction in Ca\textsuperscript{2+} diffusion may have substantial effects on CaMKII phosphorylation and the frequency dependence. For example, depending on the size of the nandomain, slow Ca\textsuperscript{2+} diffusion in a nanodomain can potentially result in a localized Ca\textsuperscript{2+} signal with sufficient strength to activate downstream cascading proteins. 

Finally, we demonstrated that holoenzyme size can affect the level of phosphorylation. We increased the size of the holoenzyme by changing the distance between neighbor subunits and found that the phosphorylation level of the network increased slightly accordingly. This is consistent with the idea that the configuration of a holoenzyme, whether compact or extended, affects the ability of CaM molecules to access CaMKII subunits~\cite{Stratton:2013el}. The extension may allow a subunit to sample a volume away from other subunits, increasing the possibility of a reaction.

Our model does not contain Thr305/Thr306 phosphorylation (few subunits would have become phosphorylated during the time frame simulated here) and also lacks some newly discovered CaMKII structural feature mechanisms, which may lead to a more complicated activation pattern of CaMKII holoenzyme subunits. For example, it has been found that there exists a compact autoinhibition state, which occurs through dimerization of adjacent subunits from top and bottom rings~\cite{Chao:2010fn}. Once Ca\textsuperscript{2+} is bound to a dimerized subunit, the dimer disassembles and the two subunits swing away from the center of the holoenzyme. Another recent study indicated that phosphorylated CaMKII subunits can undergo subunit exchange to facilitate propagating activation triggered by Ca\textsuperscript{2+}-CaM~\cite{Stratton:2014ct}. The significance of these additional features of CaMKII activation awaits future study.

One technical challenge for particle-based simulation is to handle diffusion-limited reactions, especially in the presence of highly concentrated molecules. One recent experimental study~\cite{Faas:2011fna} estimated that the N sites of CaM act very fast to bind Ca\textsuperscript{2+}, much faster than previously cited for CaM-N lobe binding kinetics in experimental or modeling studies (although see ~\cite{Mironov:2013} for a critique of these estimates). If accurate, these fast binding kinetics would place these reactions in the diffusion-limited regime, rendering traditional mass action based methods inaccurate. 
However for kinetics this fast, adequate simulation options are limited and not efficient. If using the original Smoldyn algorithm, the simulation time step would have to be considerably reduced to obtain the correct steady state~\cite{Andrews:2015}; alternatively, one might increase the geminate recombination probability~\cite{Andrews:2004fs}. Another software package using an enhanced Green's Function Algorithm~\cite{vanZon:2005jd} can handle the high concentration diffusion-limited reactions accurately, but it takes an impractically long time to simulate. If diffusion is slowed considerably in local nanodomains such that Ca\textsuperscript{2+}-CaM-CaMKII interactions become diffusion-limited, it will be necessary to develop different algorithms to handle these interactions accurately.


\section*{Methods}
\subsection*{Simulator modifications}

We expanded the molecule data structure in Smoldyn to include complexes, molecules and binding sites. A complex may contain multiple molecules and a molecule may contain multiple binding sites. Reactions are specified between binding sites. Each binding site has binary states. For example, bound is coded as 1 and unbound as 0; phosphorylated as 1 and unphosphorylated as 0. Each molecule has a vector to store the states of binding sites. All reactions are stored in a hash table with reactants and their binding states as entry keys. A hash table is a data structure that stores association arrays and allows rapid lookup. In our case, the reaction network can be considered as associations between reactants, and therefore are ideal to be implemented using a hash table. CaM is an example of a molecule with multiple binding sites. The binding reactions involving the N and C lobes of a CaM can be coded as in Fig~\ref{fig1}B. A CaMKII holoenzyme is an example of a complex composed of two 6-subunit rings. Each subunit is a molecule containing binding sites for CaM and phosphorylation. Each ring has a radius of \SI{8}{\nm} and is separated from its direct neighbors at a fixed distance of \SI{8}{\nm} (estimated from~\cite{Gaertner:2004jk}). For simplicity, we usually modeled CaMKII holoenzymes as one ring of 6 subunits.

In the original version of Smoldyn, each reaction generates new molecules and reactant molecules are removed. In our case, since one molecule can have multiple binding sites and is potentially associated with multiple partners, entirely removing a molecule is not practical because other attached molecules would also be affected. In addition, removing and generating new molecules makes it difficult to track the reaction history of a molecule. Therefore, during reactions we do not remove molecules, but merely change molecule binding states and positions. Molecules bound together physically overlap, synchronize their locations automatically and diffuse together. The diffusion coefficient is determined by the dominant molecule. For example, when Ca\textsuperscript{2+} and CaM are bound, the attached Ca\textsuperscript{2+} molecule diffuses with the CaM diffusion rate; similarly, CaM bound to a CaMKII subunit will diffuse with CaMKII.

Macromolecules usually have multiple binding sites, and sometimes these sites compete for the same ligand. For example, CaM has 4 Ca\textsuperscript{2+} binding sites. Since the N and C sites act independently, the N1 and C1 sites compete for Ca\textsuperscript{2+} and an apoCaM N0C0 can become either N1C0 or N0C1, resulting in a branching reaction scheme. Thus a decision process is needed to choose a reaction path when such a binding event occurs. 

To do this, consider the following two reactions
\begin{quote}
\begin{verbatim}
	rxn1: camN0C0 + ca <-> camN1C0
	rxn2: camN0C0 + ca <-> camN0C1
\end{verbatim}
\end{quote}

Rxn1 has a forward rate constant $k_{f1}$ in \SI{}{\per\micro\Molar\per\s} and a backward rate constant $k_{b1}$ \SI{}{\per\s}. Rxn2 has similar rate constants $k_{f2}$ and $k_{b2}$. The two reactions can be viewed together as an equivalent rxn3, which has overall kinetic rates $k_{f3}$ and $k_{b3}$. According to the law of mass action, the reactions can be written as differential equations and we can obtain
\begin{equation}
	k_{f3}=k_{f1}+k_{f2}
\end{equation}

Smoldyn uses binding radii to implement second order reactions. If two molecules are spatially separated by a distance smaller than the corresponding binding radius, then the reaction proceeds. In Smoldyn, a special algorithm is used to calculate the binding radius, which depends on the kinetic rate constant, simulation time step and total diffusion rate of reactants. In case of a branched binding scheme sharing common reactants, we first calculate a binding radius $r_3$ based on $k_{f3}$. If the distance between a molecule pair is smaller than $r_3$, binding happens. To make a reaction choice, we generate a uniformly distributed random number from 0 to 1. If the number falls in the range $(0,\frac{kf_1}{kf_3}]$, then rxn1 is chosen; instead if the number falls in the range $(\frac{kf_1}{kf_3}, 1]$, we pick rxn2. Following this approach, the network can be kept consistent with the prediction by the mass action law. 

\subsection*{Reaction network}
We focus on the interactions among Ca\textsuperscript{2+}, CaM, and CaMKII. We first used a Ca\textsuperscript{2+}-CaM network for testing to confirm that modifications to the simulator were working properly. Then we added CaMKII holoenzymes to study the reaction network in detail. We set up a cube-shaped model to represent a portion of a cell body. The cube has dimensions of \SI{1}{\um} in width and length, as well as \SI{2}{\um} in depth. The top surface of the cube represents the cell membrane, reflective to all molecules. The four sides are also reflective. The bottom surface is partially absorbing to Ca\textsuperscript{2+} ions but reflective to CaM and CaMKII. This conservation of CaM molecules and CaMKII subunits guarantees a steady state initial condition. This partial absorption is a built-in feature in the original Smoldyn to resemble unbounded diffusion~\cite{Andrews:2009dr}.


Voltage-gated Ca\textsuperscript{2+} channels (presumably L-type) are located on the top surface to provide Ca\textsuperscript{2+} influx. For simplicity, these channels are placed together at the center of the membrane. The channels open and close depending on a time-varying membrane voltage file generated from a neuron model (described below). CaMKII subunits are uniformly present at a concentration of \SI{10}{\micro\Molar}. They are also immobilized, presumably are attached to actin~\cite{Li:2016cq}. Freely diffusible CaM molecules are uniformly distributed at a concentration of \SI{5}{\micro\Molar}. This is consistent with the notion that at the resting level, freely diffusible CaM molecules are considerably limited in number compared to their binding proteins~\cite{Tran:2003fs,2008BpJ....95.6002S,LubyPhelps:1995kl}. Table S1 lists all the reactions with corresponding kinetic parameters involved in the network. Kinetic parameters are integrated from various sources as noted in Table S1 and are adjusted to satisfy microscopic reversibility. 

\subsection*{Ca\textsuperscript{2+} input conditions}

A neuron model was constructed using NEURON~\cite{Carnevale:2006iv} with a detailed morphology of a CA1 pyramidal cell and ion channel conductances to generate a voltage response at the soma to various stimulation conditions. Theta-burst stimulation (5 pulses at 100 Hz, repeated 5 times at 5 Hz or 10 Hz intervals) was applied to synapses on spines taking into account a probability of release measured in experiments~\cite{Grover:2009hb}. This stimulation activated AMPA and NMDA receptor-channels on dendritic spines causing depolarization in the dendritic tree, which propagated to the soma and initiated action potentials. Voltage profiles for 5 Hz and 10 Hz interval stimulation are shown in Fig~\ref{fig13}A,C. 

\begin{figure}[!h]
	\caption{{\bf Patterns of membrane voltage and Ca\textsuperscript{2+} influx. }
	(A)A 5 Hz action potential burst voltage file generated from the NEURON model. 
	(B)Ca\textsuperscript{2+} influx generated using the 5 Hz action potentials voltage file. Ca\textsuperscript{2+} influx is stochastic. There is a spontaneous influx in the absence of action potentials. 
	(C)10 Hz action potential burst voltage file generated from the NEURON model. 
	(D)Ca\textsuperscript{2+} influx generated using the 10 Hz action potential file. 
	(E)Accumulated Ca\textsuperscript{2+} influx for 5Hz, 10 Hz, 8 and 16 channels respectively. The total Ca\textsuperscript{2+} amount is matched between Ca\textsuperscript{2+} input files.
	}
\label{fig13}
\end{figure}

This membrane voltage output from the neuron model was used to determine Ca\textsuperscript{2+} influx through L-type channels in our reaction network model. Since the kinetics of L-type Ca\textsuperscript{2+} channels are relatively fast and their density is low, the membrane potential is little affected by their activity. Thus the neuron model and the reaction network model can be safely decoupled. L-type Ca\textsuperscript{2+} channels are modeled stochastically. They open and close in response to the voltage input. The voltage-dependent opening and closing of these channels are modeled with the Hodgkin-Huxley formalism~\cite{Tuckwell:2012tt}. The rates of channel opening and closing are functions of membrane voltage and are calculated using variables $n$ and $\tau_n$, where $n$ describes voltage-dependent activation and $\tau_n$ is the time constant of $n$ (Fig~\ref{fig14}A). The channel is assumed to obey a simple first order kinetics as follows.

\begin{figure}[!h]
	\caption{{\bf Model geometry and Ca\textsuperscript{2+} channel parameters.}
	(A)$n_{inf}(V)$ and $\tau_n(V)$ used in the voltage-gated Ca\textsuperscript{2+} channels.
	(B)The number channels N by fitting the single channel current to the GHK equation. N was found to be 16. 
	}
\label{fig14}
\end{figure}

\begin{quote}
 \verb|CaL{gate==0} <-> CaL[gate=1]|.
\end{quote}
%\begin{equation}dn(V)=\frac{dn-n_{\inf}}{dt}\end{equation}

Then the following set of equations are used to describe voltage-gated Ca\textsuperscript{2+} channel kinetics:
\begin{equation}rate_{open}(V)=\frac{n_{\inf}}{\tau_n}\end{equation}
\begin{equation}rate_{close}(V)=\frac{1-n_{\inf}}{\tau_n}\end{equation}
\begin{equation}n_{\inf}=\frac{1}{1+\exp(-\frac{V-V_{1/2}}{slope})}\end{equation}
\begin{equation}\tau_n=0.06+4\times\frac{0.75\times\sqrt{0.45\times0.55}}{\exp((V-V_{1/2})\times0.55/slope)+\exp(-(V-V_{1/2})\times0.45/slope)}\end{equation}

where $n_{\inf}$ is the steady state value of $n$, and $V_{1/2}$ and $slope$ together describe the channel activation in response to voltage. In our model, $V_{1/2}$ equals \SI{-15}{\mV} and $slope$ equals 8.

The $rate_{open}$ and $rate_{close}$ are used to calculate conditional probabilities to determine the state of a channel for the next time step in the following way
\begin{equation}P(C|O)=1-\exp(-rate_{close}(V)dt)\end{equation}
\begin{equation}P(O|C)=1-\exp(-rate_{open}(V)dt)\end{equation}
\begin{equation}P(O|O)=1-P(C|O)\end{equation}

For each channel, at a given time, a probability is calculated based on the membrane voltage to decide whether a channel opens. If it opens, a varying number of Ca\textsuperscript{2+} ions are generated. To calculate how many ions, we used the following equation~\cite{2013PNAS..11015794T} to obtain the unitary current for a single channel
\begin{equation}i_{ca}=-g(V-V_s)\frac{\exp(\frac{-(V-V_s)}{RT/zF})}{1-\exp(\frac{-(V-V_s)}{RT/zF})}\end{equation}
where $g$ is chosen as \SI{5}{\pico\siemens} and $RT/zF$ equals \SI{12}{\milli\volt}, and $V_s$ is determined as described below. A current density is calculated using GHK current equation as follows, 
\begin{equation}I=P\frac{Vz^2F^2}{RT}\frac{[Ca^{2+}]_i-[Ca^{2+}]_o\exp(-zVF/RT)}{1-\exp(-zVF/RT)}\end{equation}

where an extracellular [Ca\textsuperscript{2+}]\textsubscript{o} equals \SI{2}{\m\Molar}, an intracellular concentration [Ca\textsuperscript{2+}]\textsubscript{i} equals \SI{50}{\nano\Molar} and a maximum membrane permeability to Ca\textsuperscript{2+} P is \SI{0.241e-3}{\cm\per\s}. Since the membrane surface is \SI{1}{\square\um}, the current density is converted to a total current $I_{ghk}$ for this area. If a total of N channels are present, in this membrane surface the single channel current $i_{ca}$ equals $\frac{I_{ghk}}{N}$. By fitting the total current $I_{ghk}$ with $N\times i_{ca}$, we obtained an N of 16 and a $V_s$ of \SI{-1.91}{\milli\volt} (Fig~\ref{fig14}B). From $i_{ca}$, the number of ions entering each open channel during one time step is calculated as $\frac{i_{ca}}{2e}dt$ (Fig~\ref{fig13}B,D), where 2 is the valence of Ca\textsuperscript{2+} and $e$ is the elementary charge. To guarantee a consistent amount of total Ca\textsuperscript{2+} for a given set of 5 Hz and 10 Hz voltage files, we generated 40 trials of Ca\textsuperscript{2+} influx files for each frequency and then selected the ones with equivalent total Ca\textsuperscript{2+} influx (Fig~\ref{fig13}E).


%\section*{Conclusion}
\section*{Supporting information}

% For figure citations, please use "Fig" instead of "Figure".
% Place figure captions after the first paragraph in which they are cited.

% % Include only the SI item label in the paragraph heading. Use the \nameref{label} command to cite SI items in the text.
% \paragraph*{S1 Fig.}
% \label{S1_Fig}
% {\bf Bold the title sentence.} Add descriptive text after the title of the item (optional).

% \paragraph*{S1 File.}
% \label{S1_File}
% {\bf Lorem ipsum.}  Maecenas convallis mauris sit amet sem ultrices gravida. Etiam eget sapien nibh. Sed ac ipsum eget enim egestas ullamcorper nec euismod ligula. Curabitur fringilla pulvinar lectus consectetur pellentesque.

% \paragraph*{S1 Appendix.}
% \label{S1_Appendix}
% {\bf Lorem ipsum.} Maecenas convallis mauris sit amet sem ultrices gravida. Etiam eget sapien nibh. Sed ac ipsum eget enim egestas ullamcorper nec euismod ligula. Curabitur fringilla pulvinar lectus consectetur pellentesque.

\paragraph*{S1 Table.}
\label{S1_Table}
{\bf Kinetic parameters of all reactions.}
\paragraph*{S2 File.}
\label{S2_File}
{\bf Reaction network configuration file.}

\section*{Acknowledgments}

\nolinenumbers

% Either type in your references using
% \begin{thebibliography}{}
% \bibitem{}
% Text
% \end{thebibliography}
%
% or
%
% Compile your BiBTeX database using our plos2015.bst
% style file and paste the contents of your .bbl file
% here. See http://journals.plos.org/plosone/s/latex for 
% step-by-step instructions.
% 
\bibliography{paper}{}

\end{document}
